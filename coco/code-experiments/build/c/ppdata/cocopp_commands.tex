\providecommand{\bbobecdfcaptionsinglefunctionssingledim}[1]{
Empirical cumulative distribution of simulated (bootstrapped)
             runtimes, measured in number of $f$-evaluations,
             divided by dimension (FEvals/DIM) for the $51$ 
             targets $10^{[-8..2]}$ in dimension #1.
}
\providecommand{\cocoversion}{{\scriptsize\sffamily{}\color{Gray}Data produced with COCO v2.6.4}}
\providecommand{\numofalgs}{1}
\providecommand{\algname}{baldwinian{}}
\providecommand{\algfolder}{baldwinian/}
\providecommand{\bbobecdfcaptionallgroups}[1]{
Empirical cumulative distribution of simulated (bootstrapped)
             runtimes, measured in number of $f$-evaluations,
             divided by dimension (FEvals/DIM) for the $51$ 
             targets $10^{[-8..2]}$ for all function groups and all 
             dimensions. The aggregation over all 24 
             functions is shown in the last plot.
}
\providecommand{\bbobecdfcaptionsinglefcts}[2]{
Empirical cumulative distribution of simulated (bootstrapped) runtimes in number
             of $f$-evaluations divided by dimension (FEvals/DIM) for the 
             $51$ targets $10^{[-8..2]}$
             for functions $f_{#1}$ to $f_{#2}$ and all dimensions. 
}
\providecommand{\bbobpptablecaption}[1]{
%
        Expected runtime (\ERT) to reach given targets, measured
        in number of $f$-evaluations in #1. For each function, the \ERT\ 
        and, in braces as dispersion measure, the half difference between 10 and 
        90\%-tile of (bootstrapped) runtimes is shown for the different
        target \Df-values as shown in the top row. 
        \#succ is the number of trials that reached the last target 
        $\fopt + 10^{-8}$.
        The median number of conducted evaluations is additionally given in 
        \textit{italics}, if the target in the last column was never reached. 
        
}
\providecommand{\bbobppfigdimlegend}[1]{
%
        Scaling of runtime with dimension to reach certain target values \Df.
        Lines: expected runtime (\ERT);
        Cross (+): median runtime of successful runs to reach the most difficult
        target that was reached at least once (but not always);
        Cross ({\color{red}$\times$}): maximum number of
        $f$-evaluations in any trial. Notched boxes: interquartile range with median of simulated runs; 
        All values are divided by dimension and  
        plotted as $\log_{10}$ values versus dimension. %
        %
        Shown is the \ERT\ for fixed target precision values of $10^k$ with $k$ given
        in the legend.
        Numbers above \ERT-symbols (if appearing) indicate the number of trials
        reaching the respective target.  Horizontal lines mean linear scaling, slanted
        grid lines depict quadratic scaling.
}
\providecommand{\bbobpprldistrlegend}[1]{
%
         Empirical cumulative distribution functions (ECDF), plotting the fraction of
         trials with an outcome not larger than the respective value on the $x$-axis.
         #1%
         Left subplots: ECDF of the number of $f$-evaluations  divided by search space dimension $D$,
         to fall below $\fopt+\Df$ with $\Df=10^{k}$, where $k$ is the first value in the legend.
         The thick red line represents the most difficult target value $\fopt+ 10^{-8}$. %
         Legends indicate for each target the number of functions that were solved in at
         least one trial within the displayed budget.
         Right subplots: ECDF of the best achieved $\Df$
         for running times of $0.5D, 1.2D, 3D, 10D, 100D, 1000D,\dots$ $f$-evaluations (from right to left cycling cyan-magenta-black\dots) and final $\Df$-value (red),
         where \Df and \textsf{Df} denote the difference to the optimal function value. 
         
}
\providecommand{\bbobloglossfigurecaption}[1]{
%
        \ERT\ loss ratios (see Figure~\ref{tab:ERTloss} for details).

        Each cross ({\color{blue}$+$}) represents a single function, the line
        is the geometric mean.
        
}
\providecommand{\bbobloglosstablecaption}[1]{
%
        \ERT\ loss ratio versus the budget in number of $f$-evaluations
        divided by dimension.
        For each given budget \FEvals, the target value \ftarget\ is computed
        as the best target $f$-value reached within the
        budget by the given algorithm.
        Shown is then the \ERT\ to reach \ftarget\ for the given algorithm
        or the budget, if 
        reached a better target within the budget,
        divided by the \ERT\ of  to reach \ftarget.
        Line: geometric mean. Box-Whisker error bar: 25-75\%-ile with median
        (box), 10-90\%-ile (caps), and minimum and maximum \ERT\ loss ratio
        (points). The vertical line gives the maximal number of function evaluations
        in a single trial in this function subset. See also
        Figure~\ref{fig:ERTlogloss} for results on each function subgroup.
        
}
\providecommand{\pptablefooter}{
\end{tabular}
}
\providecommand{\pptableheader}{
\begin{tabular}{@{}c@{}|*{7}{@{}r@{}@{}l@{}}|@{}r@{}@{}l@{}}
$\Delta f$ & \multicolumn{2}{c}{1e+1} & \multicolumn{2}{c}{1e+0} & \multicolumn{2}{c}{1e-1} & \multicolumn{2}{c}{1e-2} & \multicolumn{2}{c}{1e-3} & \multicolumn{2}{c}{1e-5} & \multicolumn{2}{c}{1e-7} & \multicolumn{2}{|@{}r@{}}{\#succ}\\\hline
}
\providecommand{\bbobecdfcaptionsinglefunctionssingledim}[1]{
Empirical cumulative distribution of simulated (bootstrapped)
             runtimes, measured in number of $f$-evaluations,
             divided by dimension (FEvals/DIM) for the $51$ 
             targets $10^{[-8..2]}$ in dimension #1.
}
\providecommand{\cocoversion}{{\scriptsize\sffamily{}\color{Gray}Data produced with COCO v2.6.4}}
\providecommand{\numofalgs}{1}
\providecommand{\algname}{lamarckian{}}
\providecommand{\algfolder}{lamarckian/}
\providecommand{\bbobecdfcaptionallgroups}[1]{
Empirical cumulative distribution of simulated (bootstrapped)
             runtimes, measured in number of $f$-evaluations,
             divided by dimension (FEvals/DIM) for the $51$ 
             targets $10^{[-8..2]}$ for all function groups and all 
             dimensions. The aggregation over all 24 
             functions is shown in the last plot.
}
\providecommand{\bbobecdfcaptionsinglefcts}[2]{
Empirical cumulative distribution of simulated (bootstrapped) runtimes in number
             of $f$-evaluations divided by dimension (FEvals/DIM) for the 
             $51$ targets $10^{[-8..2]}$
             for functions $f_{#1}$ to $f_{#2}$ and all dimensions. 
}
\providecommand{\bbobpptablecaption}[1]{
%
        Expected runtime (\ERT) to reach given targets, measured
        in number of $f$-evaluations in #1. For each function, the \ERT\ 
        and, in braces as dispersion measure, the half difference between 10 and 
        90\%-tile of (bootstrapped) runtimes is shown for the different
        target \Df-values as shown in the top row. 
        \#succ is the number of trials that reached the last target 
        $\fopt + 10^{-8}$.
        The median number of conducted evaluations is additionally given in 
        \textit{italics}, if the target in the last column was never reached. 
        
}
\providecommand{\bbobppfigdimlegend}[1]{
%
        Scaling of runtime with dimension to reach certain target values \Df.
        Lines: expected runtime (\ERT);
        Cross (+): median runtime of successful runs to reach the most difficult
        target that was reached at least once (but not always);
        Cross ({\color{red}$\times$}): maximum number of
        $f$-evaluations in any trial. Notched boxes: interquartile range with median of simulated runs; 
        All values are divided by dimension and  
        plotted as $\log_{10}$ values versus dimension. %
        %
        Shown is the \ERT\ for fixed target precision values of $10^k$ with $k$ given
        in the legend.
        Numbers above \ERT-symbols (if appearing) indicate the number of trials
        reaching the respective target.  Horizontal lines mean linear scaling, slanted
        grid lines depict quadratic scaling.
}
\providecommand{\bbobpprldistrlegend}[1]{
%
         Empirical cumulative distribution functions (ECDF), plotting the fraction of
         trials with an outcome not larger than the respective value on the $x$-axis.
         #1%
         Left subplots: ECDF of the number of $f$-evaluations  divided by search space dimension $D$,
         to fall below $\fopt+\Df$ with $\Df=10^{k}$, where $k$ is the first value in the legend.
         The thick red line represents the most difficult target value $\fopt+ 10^{-8}$. %
         Legends indicate for each target the number of functions that were solved in at
         least one trial within the displayed budget.
         Right subplots: ECDF of the best achieved $\Df$
         for running times of $0.5D, 1.2D, 3D, 10D, 100D, 1000D,\dots$ $f$-evaluations (from right to left cycling cyan-magenta-black\dots) and final $\Df$-value (red),
         where \Df and \textsf{Df} denote the difference to the optimal function value. 
         
}
\providecommand{\bbobloglossfigurecaption}[1]{
%
        \ERT\ loss ratios (see Figure~\ref{tab:ERTloss} for details).

        Each cross ({\color{blue}$+$}) represents a single function, the line
        is the geometric mean.
        
}
\providecommand{\bbobloglosstablecaption}[1]{
%
        \ERT\ loss ratio versus the budget in number of $f$-evaluations
        divided by dimension.
        For each given budget \FEvals, the target value \ftarget\ is computed
        as the best target $f$-value reached within the
        budget by the given algorithm.
        Shown is then the \ERT\ to reach \ftarget\ for the given algorithm
        or the budget, if 
        reached a better target within the budget,
        divided by the \ERT\ of  to reach \ftarget.
        Line: geometric mean. Box-Whisker error bar: 25-75\%-ile with median
        (box), 10-90\%-ile (caps), and minimum and maximum \ERT\ loss ratio
        (points). The vertical line gives the maximal number of function evaluations
        in a single trial in this function subset. See also
        Figure~\ref{fig:ERTlogloss} for results on each function subgroup.
        
}
\providecommand{\pptablefooter}{
\end{tabular}
}
\providecommand{\pptableheader}{
\begin{tabular}{@{}c@{}|*{7}{@{}r@{}@{}l@{}}|@{}r@{}@{}l@{}}
$\Delta f$ & \multicolumn{2}{c}{1e+1} & \multicolumn{2}{c}{1e+0} & \multicolumn{2}{c}{1e-1} & \multicolumn{2}{c}{1e-2} & \multicolumn{2}{c}{1e-3} & \multicolumn{2}{c}{1e-5} & \multicolumn{2}{c}{1e-7} & \multicolumn{2}{|@{}r@{}}{\#succ}\\\hline
}
\providecommand{\bbobecdfcaptionsinglefunctionssingledim}[1]{
Empirical cumulative distribution of simulated (bootstrapped)
             runtimes, measured in number of $f$-evaluations,
             divided by dimension (FEvals/DIM) for the $51$ 
             targets $10^{[-8..2]}$ in dimension #1.
}
\providecommand{\cocoversion}{{\scriptsize\sffamily{}\color{Gray}Data produced with COCO v2.6.4}}
\providecommand{\numofalgs}{1}
\providecommand{\algname}{lamarckian{}}
\providecommand{\algfolder}{lamarckian/}
\providecommand{\bbobecdfcaptionallgroups}[1]{
Empirical cumulative distribution of simulated (bootstrapped)
             runtimes, measured in number of $f$-evaluations,
             divided by dimension (FEvals/DIM) for the $51$ 
             targets $10^{[-8..2]}$ for all function groups and all 
             dimensions. The aggregation over all 24 
             functions is shown in the last plot.
}
\providecommand{\bbobecdfcaptionsinglefcts}[2]{
Empirical cumulative distribution of simulated (bootstrapped) runtimes in number
             of $f$-evaluations divided by dimension (FEvals/DIM) for the 
             $51$ targets $10^{[-8..2]}$
             for functions $f_{#1}$ to $f_{#2}$ and all dimensions. 
}
\providecommand{\bbobpptablecaption}[1]{
%
        Expected runtime (\ERT) to reach given targets, measured
        in number of $f$-evaluations in #1. For each function, the \ERT\ 
        and, in braces as dispersion measure, the half difference between 10 and 
        90\%-tile of (bootstrapped) runtimes is shown for the different
        target \Df-values as shown in the top row. 
        \#succ is the number of trials that reached the last target 
        $\fopt + 10^{-8}$.
        The median number of conducted evaluations is additionally given in 
        \textit{italics}, if the target in the last column was never reached. 
        
}
\providecommand{\bbobppfigdimlegend}[1]{
%
        Scaling of runtime with dimension to reach certain target values \Df.
        Lines: expected runtime (\ERT);
        Cross (+): median runtime of successful runs to reach the most difficult
        target that was reached at least once (but not always);
        Cross ({\color{red}$\times$}): maximum number of
        $f$-evaluations in any trial. Notched boxes: interquartile range with median of simulated runs; 
        All values are divided by dimension and  
        plotted as $\log_{10}$ values versus dimension. %
        %
        Shown is the \ERT\ for fixed target precision values of $10^k$ with $k$ given
        in the legend.
        Numbers above \ERT-symbols (if appearing) indicate the number of trials
        reaching the respective target.  Horizontal lines mean linear scaling, slanted
        grid lines depict quadratic scaling.
}
\providecommand{\bbobpprldistrlegend}[1]{
%
         Empirical cumulative distribution functions (ECDF), plotting the fraction of
         trials with an outcome not larger than the respective value on the $x$-axis.
         #1%
         Left subplots: ECDF of the number of $f$-evaluations  divided by search space dimension $D$,
         to fall below $\fopt+\Df$ with $\Df=10^{k}$, where $k$ is the first value in the legend.
         The thick red line represents the most difficult target value $\fopt+ 10^{-8}$. %
         Legends indicate for each target the number of functions that were solved in at
         least one trial within the displayed budget.
         Right subplots: ECDF of the best achieved $\Df$
         for running times of $0.5D, 1.2D, 3D, 10D, 100D, 1000D,\dots$ $f$-evaluations (from right to left cycling cyan-magenta-black\dots) and final $\Df$-value (red),
         where \Df and \textsf{Df} denote the difference to the optimal function value. 
         
}
\providecommand{\bbobloglossfigurecaption}[1]{
%
        \ERT\ loss ratios (see Figure~\ref{tab:ERTloss} for details).

        Each cross ({\color{blue}$+$}) represents a single function, the line
        is the geometric mean.
        
}
\providecommand{\bbobloglosstablecaption}[1]{
%
        \ERT\ loss ratio versus the budget in number of $f$-evaluations
        divided by dimension.
        For each given budget \FEvals, the target value \ftarget\ is computed
        as the best target $f$-value reached within the
        budget by the given algorithm.
        Shown is then the \ERT\ to reach \ftarget\ for the given algorithm
        or the budget, if 
        reached a better target within the budget,
        divided by the \ERT\ of  to reach \ftarget.
        Line: geometric mean. Box-Whisker error bar: 25-75\%-ile with median
        (box), 10-90\%-ile (caps), and minimum and maximum \ERT\ loss ratio
        (points). The vertical line gives the maximal number of function evaluations
        in a single trial in this function subset. See also
        Figure~\ref{fig:ERTlogloss} for results on each function subgroup.
        
}
\providecommand{\pptablefooter}{
\end{tabular}
}
\providecommand{\pptableheader}{
\begin{tabular}{@{}c@{}|*{7}{@{}r@{}@{}l@{}}|@{}r@{}@{}l@{}}
$\Delta f$ & \multicolumn{2}{c}{1e+1} & \multicolumn{2}{c}{1e+0} & \multicolumn{2}{c}{1e-1} & \multicolumn{2}{c}{1e-2} & \multicolumn{2}{c}{1e-3} & \multicolumn{2}{c}{1e-5} & \multicolumn{2}{c}{1e-7} & \multicolumn{2}{|@{}r@{}}{\#succ}\\\hline
}
